% Created 2015-02-25 Wed 21:46
\documentclass[11pt]{article}
\usepackage[utf8]{inputenc}
\usepackage[T1]{fontenc}
\usepackage{fixltx2e}
\usepackage{graphicx}
\usepackage{longtable}
\usepackage{float}
\usepackage{wrapfig}
\usepackage{rotating}
\usepackage[normalem]{ulem}
\usepackage{amsmath}
\usepackage{textcomp}
\usepackage{marvosym}
\usepackage{wasysym}
\usepackage{amssymb}
\usepackage{hyperref}
\tolerance=1000
\date{\today}
\title{JavaCSS}
\hypersetup{
  pdfkeywords={},
  pdfsubject={},
  pdfcreator={Emacs 24.4.1 (Org mode 8.2.10)}}
\begin{document}

\maketitle
\tableofcontents

\section{JavaCSS}
\label{sec-1}


\subsection{Introduction}
\label{sec-1-1}
JavaCSS is a toolset to simplify writing Java code.

Main benefits:
\begin{itemize}
\item Automation of the output style and conventions.
\item Dependency management: import statement, pom.xml.
\end{itemize}
\subsubsection{Parser}
\label{sec-1-1-1}

JavaCSS contains an ANTLR-based Java parser. It reads Java source code, and generates an Abstract Syntax Tree (AST).
The parser and lexer are built by ANTLR from the \href{https://raw.githubusercontent.com/antlr/grammars-v4/master/java8/Java8.g4}{Java8.g4} grammar already available in ANTLR's github repository.

Looking at the grammar itself, its main entry point is the "compilationUnit" rule:
  compilationUnit
\begin{verbatim}
packageDeclaration? importDeclaration* typeDeclaration* EOF
\end{verbatim}
;

JavaCSS needs to parse whole Java files as well as certain incomplete Java snippets. Initially, the above rule seems to fit JavaCSS
requirements nicely.

\subsubsection{Generator}
\label{sec-1-1-2}

The whole process consists of:
\begin{itemize}
\item parsing Java code
\item generating an AST
\item AST processing
\item serializing the final AST
\end{itemize}

JavaCSS uses StringTemplate as generator tool. However, it currently lacks a mechanism to bind or associate templates to parts of the AST.
We'll refer to this feature as "template selectors".

\subsection{Prototype}
\label{sec-1-2}

\subsubsection{Project setup}
\label{sec-1-2-1}

As with any other regular Java project, we'll start by investing some time in preparing the tool ecosystem:

\begin{itemize}
\item Create a new repository in github.
\item Set up the folder structure expected by Maven.
\item Write the initial Maven's pom.xml
\item Create a new Jenkins job to listen to changes on the github repository.
\end{itemize}

\subsubsection{First test: Parsing an AST}
\label{sec-1-2-2}

The simplest test is simple: we want to verify the parser supports Java8 code and generates valid AST instances.
Since we just use ANTLR-provided Java grammar, the purpose of this test is a simple verification of the correctness of
the generated parser. We won't write many tests, since they don't help guiding us in the process of JavaCSS development.

Anyway, let's check if it is able to read the following Java code:

public interface Resolver
    extends Serializable \{

    public int resolve(String value);
\}

To write the test, we need to remember the API ANTLR provides for the generated parser. To build the parser instance,
we provide the text to build a Java8Lexer. Then, we instantiate a CommonTokenStream with the lexer, and pass it to the Java8Parser constructor.
Then, we call the method associated to the grammar rule we are interested in, and
get a ParseTree instance in return. Such class represents an AST.

After adding the required imports and dependencies, the test should pass.

\subsubsection{Second test: Count methods}
\label{sec-1-2-3}

What JavaCSS pursues is to aid in writing Java code, and one of such aids is freeing the developer from the task of managing
which external classes the code uses. That will eventually require us to deal with dependency management
(which library/framework a class belongs to, and how to make sure it is available when compiling or at runtime), but for now
we focus on browsing the AST to retrieve all declared types.

It's worth reviewing when such type declaration occurs in a Java source file:
\begin{itemize}
\item parameterized class/interface definitions
\item static blocks
\item instance/class attributes
\item parameterized methods
\item method returns
\item method parameters
\item local variables in methods
\item local variables in lambdas
\end{itemize}

To start simple, and to allow us to get used to traversing ASTs, method returns seem a good starting point.
But first we need to figure out how the AST itself looks like, how to distinguish a node from another, etc.
It seems we tried to be too ambitious in our test. Let's change it: instead of retrieving the list of declared
types, let's first count the methods.

The test means asking someone "how many methods are in this Java code?", but there's no one listening, yet.
Even though we don't know if it'll be a wise decision, a MethodHelper class could be handy in this context:

new MethodHelper(ast).countMethods();

However, at this point we need to dig deeper into how an AST looks like. From the grammar, we can see that rule we are
interested in is "methodDeclaration". But first, we need to learn more about ANTLR. In our context, we
can work with ParseTree objects instead of AST nodes. They are meant to be a concrete, particularized representations.
Besides that, we have three options:
\begin{enumerate}
\item Traverse the nodes recursively for each child, checking if the node corresponds to a method declaration.
\item Use a listener.
\item Use a visitor.
\end{enumerate}

The first option is not recommended, since it adds no value and it's already implemented by ANTLR-generated
classes. However, I followed it the first time, by implementing a method to check if the current
node was a method (by checking the class of node.getPayload()), and calling recursively itself for each
one of the children and incrementing the count.

However, ANTLR has anticipated our needs, and provides better options, and exported them as configuration
settings in ANTLR's Maven plugin: add <listener>true</listener> for generating the listener API, and <visitor>true</visitor> for visitors.

For this specific test, a listener-based approach fits nicely: we don't need any parsing context besides the
"methodDeclaration" rule's itself, and we don't need to tune the parsing process either.

The implementation is simple: extend Java8BaseListener to override exitMethodDeclaration(), which increments an
internal counter. Then, to retrieve the number of methods, create a ParseTreeWalker instance, call its walk(listener, node) method,
and retrieve the counter value inside the custom listener.

\subsubsection{Third test: Retrieve the types the methods return}
\label{sec-1-2-4}

Now that we know how to count the methods, we can aim higher and find the return types of the methods.
At this stage, it seems there's no real need to switch to a visitor approach. Eventually we'd probably rather skip processing
certain nodes in the tree, which we know we are not going to deal with, but not now. Or so I thought.

The new test seems to be similar to the previous one, but we are adding some variety for the types of the methods: one iteration
to build inputs with a number of methods ranging from 1 to 10, and another nested loop to provide the return types for
each of the methods, choosing randomly from a list of predefined classes. Afterwards, we check whether the types found
by our parser are the same as the original list.

The implementation is defined similarly to the previous use case: two overloaded methods. First, one that retrieves the
AST/ParseTree after parsing the input. Second, another that takes a node and uses a listener to annotate each return type.
But now, we find the first problem. Inside the exitMethodDeclarator() method, we can't retrieve the return type. We need to be
in the exitMethodHeader rule. Well, in the "result" rule, but within the "methodHeader" context. And, if the return is not "void",
within the "unannType" rule, and either within "unannPrimitiveType" or "unannReferenceType". As you can see, this approach is
going nowhere. What we do need is processing all terminal nodes which are descendant of the first "result" node, in all "methodHeader"
contexts.

Before dealing with that problem, let's review other built-in capabilities of ANTLR. It supports XPath-like expressions, so we could try
to find all terminal nodes matching "//methodHeader/result//*". 


It works perfectly for most cases, but if the type is a generic one, it contains one terminal node for the types and the '<', '>' and '?' symbols.
Using the XPath expression "\emph{/methodHeader/result//*}!typeArguments" and calling "getText()" for any non-terminal nodes doesn't work either, since
the grammar (correctly) builds different subtrees depending on the actual input and rules matched.

At this point, the only solution I see is to first ensure we are in the first occurrence of "result" within "methodHeader"; and second directly
call getText() on the rule context, regardless of the subtree therein. The latter is easy, but the former is not. How can we ensure we are processing exactly the
first "result" rule? ANTLR suggest to use labels in the grammar, but then we cannot use external, official grammars, verbatim.

Let's face it programatically. We know it's the first node once we're inside "methodHeader". There're no previous optional nodes to take care
of. By using a walker to process the first "result", and implementing a listener for that specific rule, we are done, finally.
% Emacs 24.4.1 (Org mode 8.2.10)
\end{document}
